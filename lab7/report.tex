\documentclass{article}

\usepackage[T2A]{fontenc}
\usepackage[utf8]{inputenc}
\usepackage[russian, english]{babel}
\usepackage[left=2.3cm, right=2.3cm, top=2.7cm, bottom=2.7cm, bindingoffset=0cm, margin=3in]{geometry}
\parindent 0pt
\parskip -5pt
\usepackage{amsmath}
\usepackage{amssymb}
\usepackage{amsfonts}
\usepackage{amsthm}
\usepackage{graphicx}
\usepackage[shortlabels]{enumitem}
\usepackage[all]{xy}
\usepackage[usenames]{color}
\linespread{1.5}


\usepackage{color}
\usepackage{listings}
\definecolor{mygreen}{rgb}{0,0.6,0}
\definecolor{mygray}{rgb}{0.5,0.5,0.5}
\definecolor{mymauve}{rgb}{0.58,0,0.82}
\definecolor{myorange}{rgb}{0.855,0.576,0.027}
\lstset{
    language=Octave,
    basicstyle=\ttfamily,
    frame=tb,
    extendedchars=\true,
    morecomment = [l][\itshape\color{blue}]{\%},
    keywordstyle=\color{blue},
    commentstyle=\color{mygreen},
    breakatwhitespace=false,         
    breaklines=true,  
    numbers=left,
    numbersep=-10pt,
    numberstyle=\tiny\color{mygray}, 
    showstringspaces=false,
    showtabs=false,                  
    tabsize=4,
    stringstyle=\color{myorange},
    title=\lstname,
    literate=
    {+}{{{\color{red}+}}}1
    {-}{{{\color{red}-}}}1
    {*}{{{\color{red}*}}}1
    {,}{{{\color{red},}}}1
    {=}{{{\color{red}=}}}1
    {)}{{{\color{red})}}}1
    {(}{{{\color{red}(}}}1
    {;}{{{\color{red};}}}1
    {:}{{{\color{red}:}}}1
    {[}{{{\color{red}[}}}1
    {]}{{{\color{red}]}}}1
    {>}{{{\color{red}>}}}1
}

\title{\textbf{Отчёт №7 (вар.14)} Доверительное оценивание и проверка гипотез.}
\author{Романенко Демьян, М3238}
\date{16.06.2020}

\begin{document}

    \pagenumbering{gobble}
	\maketitle
	\newpage
	\newgeometry{margin=0.8in}
	\pagenumbering{arabic}

\maketitle
    \section{Постановка задачи}
        Для случайной величины $X \thicksim B(p)$, гипотезы $H_0 : p = p_0 = 0.45$, альтернативы $H_1 : p \neq p_0$ при $n = 100$, $k_n = 37$ построить доверительный интервал для $\gamma = 0.95$ и проверить гипотезу на основании наиболее мощного критерия $\alpha = 0.05$.
        \newline
        Информация Фишера для $X \thicksim B(p) : I(p) = \frac{1}{p(1-p)}$. 
        \newline
        Величины $t_{\gamma}: \Phi_1(t_{\gamma}) = P(|\varepsilon| < t_{\gamma}) = \gamma$
        \newline
        \begin{tabular}{ | l || l | l | l | }
        \hline
        $\gamma$ & $0.9$ & $0.95$ & $0.99$  \\ \hline
        $t_{\gamma}$ & $1.65$ & $1.96$ & $2.58$ \\ 
        \hline
        \end{tabular}
    \section{Решение}
        \subsection{Построение доверительного интервала}
            ОМП {---} $\hat{\theta_n} = \hat{p_n} = \frac{k_n}{n} = \frac{37}{100} = 0.37$
            \newline
            $I(\hat{\theta_n}) = \frac{1}{0.37 * (1 - 0.37)} \approx 4.29$
            \newline
            $p_0 = 0.45 \in I_n = \left[ \hat{\theta_n} - \frac{t_{1 - \alpha}}{\sqrt{n I(\hat{\theta_n})}}; \hat{\theta_n} - \frac{t_{1 - \alpha}}{\sqrt{n I(\hat{\theta_n})}} \right] = \left[ 0.37 - \frac{1.96}{\sqrt{100 * 4.29}}; 0.37 + \frac{1.96}{\sqrt{100 * 4.29}} \right] \approx \left[ 0.27537; 0.46463 \right]$
            \newline
            Таким образом, гипотеза $H_0$ принимается.
        \subsection{Проверка гипотезы}
            Проверка гипотезы в случае двусторонней альтернативы: $\Psi_{n, \alpha}^* =
            \begin{cases}
                1, \sqrt{n I(\hat{\theta_n})}|\hat{\theta_n} - \theta_0| \geq t_{\gamma},
                \\
                0, \sqrt{n I(\hat{\theta_n})}|\hat{\theta_n} - \theta_0| < t_{\gamma};
            \end{cases}$
            \newline
            $\Psi_{n, \alpha}^* = \left[ \sqrt{n I(\hat{\theta_n})}|\hat{\theta_n} - \theta_0| \right] = \left[ \sqrt{\frac{n}{p_0(1-p_0)}}|\frac{k_n}{n} - p_0| \right] = \left[ \sqrt{\frac{100}{0.45 * (1-0.45)}}|0.37 - 0.45| \right] = \left[ 1.608 < t_{\gamma} = 1.96 \right] = 0$
            \newline
            Таким образом, гипотеза вновь принимается.
    \section{Вывод}
        В ходе лабораторной работы стало ясно, что гипотеза $H_0$ принимается.
\end{document}
 
