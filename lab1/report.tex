\documentclass{article}

\usepackage[T2A]{fontenc}
\usepackage[utf8]{inputenc}
\usepackage[russian, english]{babel}
\usepackage[left=2.3cm, right=2.3cm, top=2.7cm, bottom=2.7cm, bindingoffset=0cm, margin=3in]{geometry}
\parindent 0pt
\parskip -5pt
\usepackage{amsmath}
\usepackage{amssymb}
\usepackage{amsfonts}
\usepackage{amsthm}
\usepackage{graphicx}
\usepackage[shortlabels]{enumitem}
\usepackage[all]{xy}
\usepackage[usenames]{color}
\linespread{1.3}


\usepackage{color}
\usepackage{listings}
\definecolor{mygreen}{rgb}{0,0.6,0}
\definecolor{mygray}{rgb}{0.5,0.5,0.5}
\definecolor{mymauve}{rgb}{0.58,0,0.82}
\definecolor{myorange}{rgb}{0.855,0.576,0.027}
\lstset{
    language=Octave,
    basicstyle=\ttfamily,
    frame=tb,
    extendedchars=\true,
    morecomment = [l][\itshape\color{blue}]{\%},
    keywordstyle=\color{blue},
    commentstyle=\color{mygreen},
    breakatwhitespace=false,         
    breaklines=true,  
    numbers=left,
    numbersep=-10pt,
    numberstyle=\tiny\color{mygray}, 
    showstringspaces=false,
    showtabs=false,                  
    tabsize=4,
    stringstyle=\color{myorange},
    title=\lstname,
    literate=
    {+}{{{\color{red}+}}}1
    {-}{{{\color{red}-}}}1
    {*}{{{\color{red}*}}}1
    {,}{{{\color{red},}}}1
    {=}{{{\color{red}=}}}1
    {)}{{{\color{red})}}}1
    {(}{{{\color{red}(}}}1
    {;}{{{\color{red};}}}1
    {:}{{{\color{red}:}}}1
    {[}{{{\color{red}[}}}1
    {]}{{{\color{red}]}}}1
    {>}{{{\color{red}>}}}1
}





\title{\textbf{Отчёт №1 (вар.14)} Метод Монте-Карло}
\author{Романенко Демьян, М3238}
\date{10.03.2020}

\begin{document}

    \pagenumbering{gobble}
	\maketitle
	\newpage
	\newgeometry{margin=0.8in}
	\pagenumbering{arabic}

\maketitle
    \section{Оценка объёма}
        \subsection{Задание}
            Методом Монте-Карло оценить объем части тела $\{F(\widetilde{x}) \leq c\}$, заключенной в k-мерном кубе с ребром $[0,1]$. Функция имеет вид $F(\widetilde{x}) = f(x_1) + f(x_2) + \ldots + f(x_k)$. Для выбранной надежности $\gamma \geq 0.95$ указать асимптотическую точность оценки и построить асимптотический доверительный интервал для истинного значения объёма. 
            \newline
            
            Используя объём выборки $n = 10^4$ и $n = 10^6$, оценить скорость сходимости и показать, что доверительные интервалы пересекаются.
            \newline
            
            \paragraph{Входные данные}
                \begin{itemize}
                    \item $f(x) = x^\pi$
                    \item Размерность $k = 5$,
                    \item Параметр $c = 1.4$.
                \end{itemize}
        \subsection{Решение}
            Воспользуемся методом Монте-Карло: создадим выборку $n$ случайных векторов $x$  размерности $k$, применим к каждому из них функцию $F$ и найдём число точек, принадлежащих фигуре $(\{F(\widetilde{x}) \leq c\})$. Вычислим коэффициент $T = \frac{\gamma + 1}{2}$, и по формуле $\Delta = T\sqrt{\frac{pq}{n}}$ получим половину длины $95\%$ доверительного интервала.
        \subsection{Код программы}
\begin{lstlisting}[caption={volume.m}]
    pkg load statistics;
    
    function [res] = f (x)
      res = x.^ pi;
    endfunction
    
    function monte_carlo_volume(n)
      my_gamma = 0.95;
      k = 5;
      c = 1.4;
      X = rand(k, n);
      F_x = sum(f(X));
      my_volume = mean(F_x <= c);
      
      T = norminv((my_gamma + 1) / 2);
      my_delta = T * sqrt(my_volume * (1 - my_volume) / n);
      
      printf("N = %d\n", n);
      printf("Volume is %dL\n", my_volume);
      printf("Confidence interval: [%d, %d]\n", my_volume - my_delta, my_volume + my_delta);
      printf("Delta is %d\n\n", my_delta);
    endfunction
    
    monte_carlo_volume(10000);
    monte_carlo_volume(1000000);
\end{lstlisting}
        \subsection{Результаты}
\begin{verbatim}
N = 10000
Volume is 0.6403L
Confidence interval: [0.630894, 0.649706]
Delta is 0.00940611

N = 1000000
Volume is 0.644745L
Confidence interval: [0.643807, 0.645683]
Delta is 0.00093802
\end{verbatim}
        \subsection{Вывод}
            Скорость схождения пропорциональна $\sqrt{n}$, поскольку при увеличении числа итераций в $100$ раз ширина доверительного интервала уменьшилось в $10$ раз. Границы доверительного интервала вложены в друг друга при увелечении выборки с $n = 10^4$ до $n = 10^6$.
    \section{Оценка значения интегралов}
        \subsection{Задание}
            Аналогично построить оценку интегралов (представить интеграл как математическое ожидание функции, зависящей от случайной величины с известной плотностью) и для выбранной надежности $\gamma \geq 0.95$ указать асимптотическую точность оценивания и построить асимптотический доверительный интервал для истинного значения интеграла.
            \paragraph{Входные данные}
                \begin{itemize}
                    \item $f(x) = x^\pi$
                    \item Размерность $k = 5$,
                    \item Параметр $c = 1.4$.
                    \item Первый интеграл $\int_{-\infty}^{\infty} |x|\,\exp{(-(x-2)^2/3)}\,dx$
                    \item Второй интеграл $\int_{2}^{7} \sqrt[4]{2+x^2}\,dx$
                \end{itemize}
        \subsection{Решение}
            Для сравнения с результатом явного вычисления интеграла воспользуемся функцией $quad$. 
            \subsubsection{Первый интеграл}
                Первый интеграл приводится к виду нормального распределения функции $g(x) = |x|$ с параметрами $\sigma = \sqrt{\frac{3}{2}}$ и $\mu = 2$:
                \newline
                $\displaystyle \int\limits_{-\infty}^{\infty} |x|\,\exp{(-\frac{(x-2)^2}{3})}\,dx = \sigma\sqrt{2\pi}(\frac{1}{\sigma\sqrt{2\pi}} \int\limits_{-\infty}^{\infty} g(x)\,\exp{(-\frac{(x-\mu)^2}{2\sigma^2})\,dx}) = \sqrt{3\pi}(\frac{1}{\sqrt{3\pi}} \int\limits_{-\infty}^{\infty} g(x)\,\exp{(-\frac{(x-2)^2}{2(\sqrt{\frac{3}{2}})^2})}\,dx) =$ 
                \newline
                $= E_{N(\mu, \sigma)}(\sigma \sqrt{2\pi}g(x)) = E_{N(2, \sqrt{\frac{3}{2}})}( \sqrt{3\pi}|x|)$
            \subsubsection{Второй интеграл}
                Первый интеграл приводится к виду равномерного распределения функции $g(x) = \sqrt[4]{2+x^2}$ с параметрами $a = 2$ и $b = 7$:
                \newline
                $\displaystyle \int_{2}^{7} \sqrt[4]{2+x^2}\,dx = (b - a)(\frac{1}{b - a} \int_{a}^{b} g(x)\,dx) = 5(\frac{1}{5} \int_{2}^{7} g(x)\,dx) = E_{V(a,b)}((b - a)g(x)) = E_{V(2,7)}(5\sqrt[4]{2+x^2})$
        \subsection{Код программы}
            \subsubsection{Первый интеграл}
\begin{lstlisting}[caption={a.m}]
    pkg load statistics;

    function [res] = g(x)
      res = abs(x);
    endfunction
    
    function [res] = g1(x)
      res = sqrt(3 * pi) * g(x);
    endfunction
    
    function [res] = f(x)
      res = g(x) * exp(-(x - 2).^2 / 3);
    endfunction
    
    function monte_carlo_a(n)
      my_gamma = 0.95;
      
      u = 2;
      sigma = sqrt(3 / 2);
      
      Q = norminv((my_gamma + 1) / 2);
      X = normrnd(u, sigma, 1, n);
      
      F_x = g1(X);
      V = mean(F_x);
      my_delta = (std(F_x) * Q) / sqrt(n);
      
      printf("N = %d\n", n);
      printf("Value is %d\n", V);
      printf("Confidence interval: [%d, %d]\n", V - my_delta, V + my_delta);
      printf("Delta is %d\n\n", my_delta);
    endfunction
    
    printf("Sample answer = %d\n\n", quad(@f, -inf, inf));
    monte_carlo_a(10000);
    monte_carlo_a(1000000);
\end{lstlisting}
            \subsubsection{Второй интеграл}
\begin{lstlisting}[caption={b.m}]
    pkg load statistics;

    function res = f(x)
      res = (x.^2 + 2).^(1/4);
    endfunction

    
    function monte_carlo_b(n)
      my_gamma = 0.95;
      
      L = 2;
      R = 7;
      
      Q = norminv((my_gamma + 1) / 2);
      X = unifrnd(L, R, 1, n);
      
      F_x = f(X) * (R - L);
      V = mean(F_x);
      my_delta = (std(F_x) * Q) / sqrt(n);
      
      printf("N = %d\n", n);
      printf("Value is %d\n", V);
      printf("Confidence interval: [%d, %d]\n", V - my_delta, V + my_delta);
      printf("Delta is %d\n\n", my_delta);
    endfunction
    
    printf("Sample answer = %d\n\n", quad(@f, 2, 7));
    monte_carlo_b(10000);
    monte_carlo_b(1000000);
\end{lstlisting}
        \subsection{Результаты}
            \subsubsection{Первый интеграл}
\begin{verbatim}
Sample answer = 6.30159

N = 10000
Value is 6.28071
Confidence interval: [6.21294, 6.34849]
Delta is 0.0677757

N = 1000000
Value is 6.30446
Confidence interval: [6.29764, 6.31129]
Delta is 0.00682503
\end{verbatim}
            \subsubsection{Первый интеграл}
\begin{verbatim}
̀Sample answer = 10.7685

N = 10000
Value is 10.7825
Confidence interval: [10.7511, 10.8139]
Delta is 0.0314257

N = 1000000
Value is 10.7701
Confidence interval: [10.7669, 10.7732]
Delta is 0.00312935
\end{verbatim}
        \subsection{Вывод}
            Границы доверительного интервала вложены в друг друга при увелечении выборки с $n = 10^4$ до $n = 10^6$.
            Явно вычесленное (при помощи функции $quad$) значение интеграла лежит в границах доверительного интер\-вала и отличается от значения, полученного методом Монте-Карло, на $2.9 \cdot 10^{-3}$ для первого интеграла и на $1.6 \cdot 10^{-3}$ для второго. Скорость схождения пропорциональна $\sqrt{n}$, поскольку при увеличении числа итераций в $100$ раз ширина доверительного интервала уменьшилось в $10$ раз.
\end{document}
