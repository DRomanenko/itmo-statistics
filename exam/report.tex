\documentclass{article}

\usepackage[T2A]{fontenc}
\usepackage[utf8]{inputenc}
\usepackage[russian, english]{babel}
\usepackage[left=2.3cm, right=2.3cm, top=2.7cm, bottom=2.7cm, bindingoffset=0cm, margin=3in]{geometry}
\parindent 0pt
\parskip -5pt
\usepackage{amsmath}
\usepackage{amssymb}
\usepackage{amsfonts}
\usepackage{amsthm}
\usepackage{graphicx}
\usepackage[shortlabels]{enumitem}
\usepackage[all]{xy}
\usepackage[usenames]{color}
\linespread{1.5}


\usepackage{color}
\usepackage{listings}
\definecolor{mygreen}{rgb}{0,0.6,0}
\definecolor{mygray}{rgb}{0.5,0.5,0.5}
\definecolor{mymauve}{rgb}{0.58,0,0.82}
\definecolor{myorange}{rgb}{0.855,0.576,0.027}
\lstset{
    language=Octave,
    basicstyle=\ttfamily,
    frame=tb,
    extendedchars=\true,
    morecomment = [l][\itshape\color{blue}]{\%},
    keywordstyle=\color{blue},
    commentstyle=\color{mygreen},
    breakatwhitespace=false,         
    breaklines=true,  
    numbers=left,
    numbersep=-10pt,
    numberstyle=\tiny\color{mygray}, 
    showstringspaces=false,
    showtabs=false,                  
    tabsize=4,
    stringstyle=\color{myorange},
    title=\lstname,
    literate=
    {+}{{{\color{red}+}}}1
    {-}{{{\color{red}-}}}1
    {*}{{{\color{red}*}}}1
    {,}{{{\color{red},}}}1
    {=}{{{\color{red}=}}}1
    {)}{{{\color{red})}}}1
    {(}{{{\color{red}(}}}1
    {;}{{{\color{red};}}}1
    {:}{{{\color{red}:}}}1
    {[}{{{\color{red}[}}}1
    {]}{{{\color{red}]}}}1
    {>}{{{\color{red}>}}}1
}

\title{\textbf{Экзамен, 5-ый билет.}}
\author{Романенко Демьян, М3238}
\date{22.06.2020}

\begin{document}

    \pagenumbering{gobble}
	\maketitle
	\newpage
	\newgeometry{margin=0.8in}
	\pagenumbering{arabic}

\maketitle
    \section{Билет}
        \begin{enumerate}
            \item Ошибки первого и второго рода и их вероятности как критерий качества критерия (теста) проверки гипотез. Подход Неймана-Пирсона.
            \item Свойства ЭФР в целом. Расстояние Колмогорова, Смирнова. Теоремы Гливенко-Кантелли, Колмогорова, Мизеса - Смирнова. Построение доверительной полосы для функции распределения.
            \item Пусть простейший процесс имеет вид $X(t) = Y t + c$, $Y \sim U(a, b)$. Найти корреляционную функцию и нормированную корреляционную функцию этого процесса.
        \end{enumerate}
    \section{Ответы}
    \subsection{Первый вопрос}
    \subsubsection{Ошибки первого и второго рода и их вероятности как критерий качества критерия (теста) проверки гипотез.}
        Пусть $H_0: \theta \in H_0$ -- гипотеза, $H_1: H_0 \subset \Theta, H_1 = \bar{H_0}$ -- альтернатива, $\varphi: \mathcal{X} \xrightarrow{}$ $\{0, 1\}$ -- тест.
        \newline
        \textit{Ошибкой I рода} называют отклонение основной гипотезы, в то время как она была справедлива.
        \newline
        \textit{Вероятность ошибок I рода} теста $\varphi$ называют $\alpha$ такую, что: $\alpha(\varphi, \theta) := P_{\theta}(\mathcal{X}_{n, 1})$, $\theta \in \Theta_{H_0}$.
        \newline
        \textit{Ошибкой II рода} называют принятие основной гипотезы, в то время как она не была справедлива.
        \newline
        \textit{Вероятность ошибок II рода} теста $\varphi$ называют $\beta$ такую, что: $\beta(\varphi, \theta) := P_{\theta}(\mathcal{X}_{n, 0})$, $\theta \in \Theta_{H_1}$
        \newline
        \textit{Уровнем значимости теста} называют верхнюю границу вероятности ошибки I рода по всем возможным наблюдаемым значениям неизвестных параметров, отвечающих основной гипотезе: $\alpha(\varphi) := \sup\limits_{\theta \in \Theta_{H_0}} \alpha(\varphi, \theta)$
        \textit{Мощностью теста} называют следующую величину: $\gamma(\varphi, \theta) := 1 - \beta(\varphi, \theta)$
    \subsubsection{Подход Неймана-Пирсона.}  
        Будем искать лучшую оценку в классе оценок с зафиксированным максимальным значением уровня значимости. Зафиксируем $\alpha \in (0, 1)$ (обычно выбирают малое значение, близкое к нулю). Будем считать это значение минимальной допустимой величиной ошибки I рода (\textit{допустимый уровень значимости}).
        Рассмотрим множество всех тестов таких, что: $\bar{\Phi}_\alpha = \{\varphi = \varphi(x) \mid \alpha(\varphi) \leqslant \alpha\}$. Среди этих тестов выбирается тест с минимальным значением $\beta$. В асимптотических задачах ограничения накладываются на предельные значения.
    \subsection{Второй вопрос}
    \subsubsection{Свойства ЭФР в целом.}
    \textit{Эмпирической функцией распределения (ЭФР)} в точке $t \in \mathbb{R}$ называют следующую оценку функции распределения генеральной совокупности: $F_n(t) = \frac{1}{n} \sum_{i = 1}^{n} 1_{(-\infty, t)}.$ Иными словами, значение ЭФР в точке $t$ равно отношению числа наблюдений, меньших $t$, к их общему числу $n$.
    \paragraph{Свойства ЭФР:}
    \begin{enumerate}
        \item ЭФР кусочно-постоянна;
        \item Скачки ЭФР имеют вид $\frac{k}{n}$ для некоторого $k \in (1; n)$;
        \item Область принимаемых значений: $[0; 1]$;
        \item $F_n(t)$ является состоятельной оценкой: $F_n(t_0) = \bar{\xi}_n: F_n(t_0) \xrightarrow[p = 1]{} F_x$;
        \item $F_n(t)$ является асимптотически нормальной оценкой;
        \item Частота может служить как оценка функции распределения генеральной
            совокупности. При фиксированном $t = t_0$: $F_x(t_0) \approx F_n(t_0) = \frac{\xi_1 + \ldots + \xi_n}{n} = \frac{k_n}{n}~$ \text{-- частота}.
    \end{enumerate}
    \subsubsection{Расстояние Колмогорова, Смирнова.}
        \textit{Расстояние Колмогорова}: $\rho_{\infty} (F_n, F_x) = \sup\limits_{t}|F_n(t) - F_x(t)|$
        \newline
        \textit{Расстояние Смирнова}: $\rho^{2}_{2} (F_n, F_x) = \int\limits_{\mathbb{R}}(F_n(t) - F_x(t))^2$ d$F_x(t).$
    \subsubsection{Теоремы Гливенко-Кантелли, Колмогорова, Мизеса - Смирнова}
    \paragraph{Теорема Гливенко-Кантелли}
        Пусть $\mathcal{F}$ -- множество функций распределения. Тогда $\forall F_x(t) \in \mathcal{F}$ c вероятностью 1 справедливо предельное неравенство: $\rho_{\infty} (F_n, F_x) \xrightarrow[n \to \infty]{} 0.$ То же верно для $\rho_2$, так как $\rho_2 \leqslant \rho_{\infty}$. $F_n(t)$ -- состоятельная оценка $F_x(t)$ в расстояниях Колмогорова и Смирнова.
    \paragraph{Теорема Колмогорова}
        Пусть $\mathcal{F}_c$ -- множество всех непрерывных функций распределения, $F_x \in \mathcal{F}_c$. Тогда
        \newline
        $P_{n, F}(\sqrt{n} \rho_{\infty} (F_n, F_x) < u) \xrightarrow[n \to \infty]{} \mathcal{K}(u) =$ $\begin{cases}
            0,~ u = 0,\\
            \sum\limits_{j = -\infty}^{+\infty} (-1)^j e^{-2 (ju)^2},~ u > 0.
        \end{cases}$
    \paragraph{Теорема Мизеса - Смирнова}
        $P_{n, F}(\sqrt{n} \rho^{2}_{2} (F_n, F_x) < u) \xrightarrow[n \to \infty]{} \mathcal{S}(u)$, где $\mathcal{S}(u)$ есть функция распределения следующей случайной величины: $\mathcal{U} = \sum\limits_{j = 1}^{\infty} \frac{\xi^2_j}{j^2 \pi^2},~ \xi_j \sim N(0, 1),~ \xi_j \text{независимые}.$
    \subsubsection{Построение доверительной полосы для функции распределения.}
        Используя теорему Колмогорова, можно построить доверительную полосу для неизвестной функции распределения.
        \newline
        \textit{Доверительной полосой} $\gamma$ называют часть плоскости, в которую с надежностью $\gamma$ попадает функция распределения генеральной совокупности.
    \paragraph{Теорема}
        Доверительная полоса задаётся функциями:
        $F^{-}_n(t) = \max\left(0, F_n(t) - \frac{u_{\gamma}}{\sqrt{n}}\right)$, $F^{+}_n(t) = \min\left(1, F_n(t) + \frac{u_{\gamma}}{\sqrt{n}}\right)$, где $u_{\gamma}$ определяется из условия $\mathcal{K}(u_{\gamma}) = \gamma$.
    \newline    
    $\square$    
    \newline 
    Достаточно проверить, что $P_x(F^{-}_n(t) \leqslant F_x(t) \leqslant F^{+}_n(t))
    \xrightarrow[n \to \infty]{} \gamma$:
    \newline
    $P_x(F^{-}_n(t) \leqslant F_x(t) \leqslant F^{+}_n(t)) = P_x\left(F_x(t) - \frac{u_{\gamma}}{\sqrt{n}} \leqslant F_x(t) \leqslant F_n(t) + \frac{u_{\gamma}}{\sqrt{n}}\right) = P_x\left(\sqrt{n} |F_x(t) - F_n(t)| \leqslant u_{\gamma}\right) =$ 
    \newline
    $= P_x\left(\sqrt{n} \sup\limits_{t \in \mathbb{R}} |F_x(t) - F_n(t)| \leqslant u_{\gamma}\right) \xrightarrow[n \to \infty]{} \mathcal{K}(u_{\gamma}) = \gamma$
    \newline
    \begin{flushright}
        $\blacksquare$
    \end{flushright}
    \section{Задача}
    \paragraph{Формулировка}
        Пусть простейший процесс имеет вид $X(t) = Y t + c$, $Y \sim U(a, b)$. Найти корреляционную функцию и нормированную корреляционную функцию этого процесса.
    \paragraph{Решение}
        \begin{enumerate}
            \item Найдём матожидание процесса: $E_X(t) = E(Y t + c) = t E(y) + c = t \frac{a + b}{2} + c$;
            \item Найдем вид центрированного сечения: $\tilde{X}(t) = X(t) + E_X(T) = Y t + c - t \frac{a + b}{2}$;
            \item Найдём саму корреляционную функцию этого процесса: $K_X(t, t') = E(\tilde{X}(t) * \tilde{X}(t')) = t t' E(Y - \frac{a + b}{2})^2 =$
            \newline
            $= t t' D(Y) = t t' \frac{(b - a)^2}{12}$;
            \item Найдём дисперсию сечений: $D_X(t) = D(Y t + c) = t^2 D(Y) = t^2 \frac{(b - a)^2}{2};$
            \item Найдём саму нормированную корреляционную функцию этого процесса: $\rho_X(t, t') = \frac{K_X(t, t')}{\sigma_X(t) \sigma_X(t')} =$ 
            \newline
            $= t t' \frac{(b - a)^2}{12} \sqrt{\frac{12}{t^2 (b - a)^2}} \sqrt{\frac{12}{t'^2(b - a)^2}} = 1$;
        \end{enumerate}
        Проверка: из определения $X(t)$ зависимость между сечениями должна быть линейной, какой она и является по результату вычислений.
\end{document} 
